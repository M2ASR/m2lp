\title{M2ASR NP Package Documentation}
\author{
        Dong Wang \\
                CSLT, Tsinghua University\\
        Beijing 100084, \underline{China}
            \and
        xx\\
        xxxxxx University\\
        xxxx, \underline{xxx}
}
\date{\today}

\documentclass[12pt]{article}

\begin{document}
\maketitle

\begin{abstract}

This is a document for  M2LP package. 

\end{abstract}

\section{M2ASR introduction}

M2LP is part of the M2ASR projected, and it is an open-source package for multi-minority langauge processing tasks.

Multi-Minority Automatic Speech Recognition (M2ASR) is an NSFC-supported project, with the aim of providing speech recognition service for minority nations of China. We are planning to investigate, develope and architect the ASR service within 5 years. We will pulish all the tools, data, model, and service for free.

M2ASR is a multi-party project involing CSLt@Tsinghua Unviersity, Xinjiang University, Northwest University for Nations (NUW). M2LP provides the basic functions for langauge processing, including text normalization, morphology segmentation, langauge modeling, and all the related resources, functions, and tools.


\section{M2LP pacakge}

The basic design principle is reusable and modulation. The functions will be implemented as functions written by c/c++, python or perl, and the recipe will be a shell script that integrating modules by piple line.


DIRECORY

\begin{itemize}
\item rc : multilingual langauge resource
\item src : source code for LP
\item md : model released
\end{itemize}

%\bibliographystyle{abbrv}
%\bibliography{main}

\end{document}
